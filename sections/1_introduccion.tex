\section{Introducción}

Las vibraciones mecánicas en edificios y otras estructuras constituyen una parte fundamental del área de estudio de la ingeniería civil. En particular, un caso de estudio frecuente en este campo es el análisis de las vibraciones en puentes. Es necesario realizar mediciones y estudios pertinentes de tanto dichas estructuras como los entornos sobre los que se encuentran con el fin de prevenir eventos catastróficos. El análisis de fenómenos tales como la resonancia son esenciales debido a sus implicaciones sobre la estabilidad de una estructura.

Un caso de accidente particularmente conocido es el del puente Yanango en Junín, ocurrido en noviembre de 2005. Accidentes de esta naturaleza demuestran la importancia de analizar las circunstancias a la que dichas estructuras se encuentran sometidas, ya sean propiamente estructurales o externas, como podrían ser las condiciones topográficas e hidrológicas \citep{mayhua}.

Para construir modelos físicos, tales como las vibraciones en puentes y estructuras afines, las ecuaciones diferenciales surgen como una herramienta fundamental. Nuestro principal objetivo será describir las vibraciones en puentes mediante la construcción de modelos físicos y matemáticos de ecuaciones diferenciales. Si bien varios de los aspectos a tratar ya han sido estudiados previamente por otros autores, la viabilidad de la propuesta se basa en tanto la generalización de algunos de estos conceptos como el planteamiento y estudio de un caso particular del cual se hará desarrollo posteriormente.